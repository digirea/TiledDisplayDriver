\documentclass[a4paper,10pt,oneside]{jsbook}
%
\usepackage{amsmath,amssymb,bm}
\usepackage{bm}
\usepackage[dvipdfmx]{graphicx}
\usepackage{ascmac}
\usepackage{makeidx}
\usepackage{txfonts}
\usepackage{indentfirst}
\usepackage{booktabs}
\usepackage{tabularx}
\usepackage{comment}
\AtBeginDvi{\special {pdf:tounicode EUC-UCS2}}
\usepackage[dvipdfmx, setpagesize=false, bookmarks=true, bookmarksnumbered=true]{hyperref}
%
\makeindex
%
\newcommand{\diff}{\mathrm{d}}            %微分記号
\newcommand{\divergence}{\mathrm{div}\,}  %ダイバージェンス
\newcommand{\grad}{\mathrm{grad}\,}       %グラディエント
\newcommand{\rot}{\mathrm{rot}\,}         %ローテーション
%
\setlength{\textwidth}{\fullwidth}
\setlength{\textheight}{44\baselineskip}
\addtolength{\textheight}{\topskip}
\setlength{\voffset}{-0.6in}
%

\begin{document}

%%%%%%%%%%%%%%%%%%%%%%%%%%%%%%%%%%%%%%%%%%%%%%%%%%%%%
% 表紙
\begin{titlepage}
\noindent
独立行政法人 理化学研究所 御中
\begin{center}
	\vspace{8cm}
	{\Huge \textbf{協調ワークスペースドライバと\\協調フレームワークのプロトタイプ整備} } \\
	\vspace{1cm}
	{\Huge \textbf{利用外部ライブラリ リスト}} \\
	\vspace{10cm}
	{\Large \textbf{2015年3月26日}} \\
	\vspace{0.5cm}
	{\Large \textbf{株式会社イマジカ デジタルスケープ}}
\end{center}
\end{titlepage}

%%%%%%%%%%%%%%%%%%%%%%%%%%%%%%%%%%%%%%%%%%%%%%%%%%%%%
% 目次
\tableofcontents

%%%%%%%%%%%%%%%%%%%%%%%%%%%%%%%%%%%%%%%%%%%%%%%%%%%%%
% 本文
%%%%%%%%%%%%%%%%%%%%%%%%%%%%%%%%%%%%%%%%%%%%%%%%%%%%%
\chapter{使用モジュールおよび権利処理}
協調ワークスペースドライバと協調フレームワークのプロトタイプ整備において,利用した既存モジュールとその権利処理について記載する.
各利用モジュールの名称,権利者,権利処理内容,URLを表\ref{ops}に記載する.

\begin{table}[htbp]
\begin{center}
\caption{利用モジュールおよび権利処理一覧}
\label{ops}
\begin{tabular}{|l|l|l|l|}
\hline
名称 & 権利者 & 権利処理 & URL \\
\hline
\hline
node.js & \shortstack[l]{Joyent, Inc. and \\ other Node contributors} & MIT License & https://nodejs.org/ \\
\hline
socket.io & Automattic & MIT License & http://socket.io/  \\
\hline
redis & Salvatore Sanfilippo & \shortstack[l]{three-clause\\BSD license} & http://redis.io/ \\
\hline
Qatrix & Angel Lai &  MIT License & http://qatrix.com/\\
\hline
PhantomJS & Ariya Hidayat & \shortstack[l]{three-clause\\BSD license}  & http://phantomjs.org/\\
\hline
Qt & Digia Plc  & LGPL 2.1 & http://qt-project.org/\\
\hline
Mongoose & Sergey Lyubka  & MIT License & https://github.com/cesanta/mongoose\\
\hline
Breakpad & Google Inc & \shortstack[l]{three-clause\\BSD license} & \shortstack[l]{http://code.google.com/\\p/google-breakpad/}\\
\hline
OpenSSL & The OpenSSL Project & \shortstack[l]{OpenSSL License\\SSLeay License} & http://www.openssl.org/\\
\hline
Linenoise & \shortstack[l]{Salvatore Sanfilippo\\Pieter Noordhuis} & \shortstack[l]{two-clause\\BSD license} & https://github.com/tadmarshall/linenoise\\
\hline
QCommandLine & Corentin Chary  & LGPL 2.1 & http://www.webkit.org/\\
\hline
wkhtmlpdf & wkhtmltopdf authors & LGPL 2.1 & http://code.google.com/p/wkhtmltopdf/\\
\hline
Webkit & Apple Inc  & LGPL 2.1 & http://www.webkit.org/\\
\hline
phantomjs & The Obvious Corporation & Apache License & https://github.com/Medium/phantomjs\\
\hline
image-size & Aditya Yadav & MIT License & http://netroy.in\\
\hline
WebSocket-Node & Worlize & Apache 2.0 License & \shortstack[l]{https://github.com/theturtle32/\\WebSocket-Node}\\
\hline
Bootstrap & Twitter, Inc. &  MIT License & https://github.com/twbs/bootstrap\\
\hline
Bootstrap without jQuery & Daniel Davis &  MIT License & \shortstack[l]{https://github.com/tagawa/\\bootstrap-without-jquery}\\
\hline

\end{tabular}
\end{center}
\end{table}

\end{document}
