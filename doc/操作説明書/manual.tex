\documentclass[a4paper,10pt,oneside]{jsbook}
%
\usepackage{amsmath,amssymb,bm}
\usepackage{bm}
\usepackage[dvipdfmx]{graphicx}
\usepackage{ascmac}
\usepackage{makeidx}
\usepackage{txfonts}
\usepackage{indentfirst}
\usepackage{booktabs}
\usepackage{tabularx}
\usepackage{comment}
\AtBeginDvi{\special {pdf:tounicode EUC-UCS2}}
\usepackage[dvipdfmx, setpagesize=false, bookmarks=true, bookmarksnumbered=true]{hyperref}
\usepackage{nameref}
\usepackage{url}
%
\makeindex
%
\newcommand{\diff}{\mathrm{d}}            %微分記号
\newcommand{\divergence}{\mathrm{div}\,}  %ダイバージェンス
\newcommand{\grad}{\mathrm{grad}\,}       %グラディエント
\newcommand{\rot}{\mathrm{rot}\,}         %ローテーション
%
\setlength{\textwidth}{\fullwidth}
\setlength{\textheight}{44\baselineskip}
\addtolength{\textheight}{\topskip}
\setlength{\voffset}{-0.6in}
%

\begin{document}

%%%%%%%%%%%%%%%%%%%%%%%%%%%%%%%%%%%%%%%%%%%%%%%%%%%%%
% 表紙
\begin{titlepage}
\noindent
独立行政法人 理化学研究所 御中
\begin{center}
	\vspace{8cm}
	{\Huge \textbf{協調ワークスペースドライバと}} \\
	\vspace{1cm}
	{\Huge \textbf{協調動作フレームワークのプロトタイプ}} \\
	\vspace{1cm}
	{\Huge \textbf{操作説明書}} \\
	\vspace{10cm}
	{\Large \textbf{2015年3月26日}} \\
	\vspace{0.5cm}
	{\Large \textbf{株式会社イマジカ デジタルスケープ}}
\end{center}
\end{titlepage}



%%%%%%%%%%%%%%%%%%%%%%%%%%%%%%%%%%%%%%%%%%%%%%%%%%%%%
% 目次
\tableofcontents

%%%%%%%%%%%%%%%%%%%%%%%%%%%%%%%%%%%%%%%%%%%%%%%%%%%%%
% 本文
%%%%%%%%%%%%%%%%%%%%%%%%%%%%%%%%%%%%%%%%%%%%%%%%%%%%%
\chapter{はじめに}
本書では協調ワークスペースドライバと協調動作フレームワークのプロトタイプの操作方法について解説します.\\

\section{動作環境とインストール}
以下の環境で動作します.

\begin{tabbing}
0123\=01234567890123\=0123456789\kill
\> OS \> : Linux, Windows(Vista,7,8), MacOSX \\
\> Webブラウザ \> : Mozilla Firefox 15.x, Google Chrome 21.x, Apple Safari 6.x, Windows Internet Explorer 10.x 
\end{tabbing}

\section{インストール}

\subsection{Node.jsのインストール}
ポータルGUIの動作にはNode.jsのインストールが必要です.\\
Node.jsの公式サイト(\verb+http://nodejs.org/+)からNode.js本体をダウンロードし,インストールします.

\subsection{プログラムの展開}

TODO:

\subsection{Node.jsサブモジュールのインストール}
アプリケーションを展開したディレクトリに,
ポータルGUIで利用しているNode.jsの必要なサードパーティモジュールのインストールを行います.

TODO:

\begin{verbatim}
   $cd bin
   $sh install.sh 
   (Windows版は install.bat)
\end{verbatim}

\newpage

\section{起動}
起動スクリプトを実行するとポータルGUIサーバーが起動します.
\begin{verbatim}
   $sh run.sh
   (Windows版は run.bat)
\end{verbatim}

\section{コントローラへアクセス}
ポータルGUIは,Webブラウザのアドレス欄に「http://localhost:8080」と入力することでアクセス出来ます.\\

TODO:

\chapter{ホーム画面}
\section{概要}
ホーム画面では,以下の操作を行う事が出来ます.

\begin{itemize}
\item コントローラ画面へ
\item ディスプレイ画面へ
\end{itemize}

\section{操作}

TODO:

\begin{figure}[htbp]
	\begin{center}
		\includegraphics[width=11.5cm]{image/home.png}
	\end{center}
	\caption{ホーム画面}
	\label{fig:home}
\end{figure}


\chapter{動作環境とインストール}
以下の環境で動作します.\\

OS : Linux, Windows(Vista,7,8), MacOSX
Web ブラウザ: Mozilla Firefox 15.x, Google Chrome 21.x, Apple Safari 6.x

\chapter{アプリケーションの展開}
XXXXXX.zipを解凍すると、以下の構成でファイルが作成されます.\\

\begin{tabbing}
0123\=01234567890123\=0123456789\kill
\>bin        \> : 実行スクリプトフォルダ\\
\>client     \> : TDDクライアントアプリケーションフォルダ\\
\>doc        \> : ドキュメントフォルダ\\
\>redis      \> : redisアプリケーションフォルダ\\
\>server     \> : TDDサーバーアプリケーションフォルダ\\
\>package.json
\end{tabbing}

TiledDisplayDriverの起動にはbinフォルダに格納されているスクリプトを使用します.\\

\chapter{アプリケーションのインストール方法}

\subsection{Mac/Linuxの場合}
bin配下の以下のシェルスクリプトを実行します.\\

\begin{verbatim}
   $cd bin
   $sh install.sh
\end{verbatim}

\subsection{Windowsの場合}
bin配下の以下のファイルを実行します.\\
\begin{verbatim}
   >cd bin
   >install.bat
\end{verbatim}



\chapter{アプリケーションの起動方法}


\subsection{Mac/Linuxの場合}
bin配下の以下のシェルスクリプトを実行します.\\
./run.sh



\subsection{Windowsの場合}
bin配下の以下のファイルを実行します.\\
\begin{verbatim}
   >cd bin
   >run.bat
\end{verbatim}


※
Windowsの場合、仮想メモリを0KByteにしていると、
redisが正常に起動しない場合があります.\\
その場合は一時的に仮想メモリを有効にしてご利用ください.\\


\chapter{アプリケーションの終了方法}

以下2点にて終了させます.\\

\subsection{1.サーバープログラムの終了}\
run.sh(bat)を起動したterminalをCTRL+Cで終了するか、
serverプログラムをkillします.\\


\subsection{2.redisの終了}
redisが起動しているterminalを終了させます.\\
また、プロセスとして起動している場合は、プロセスをpsコマンドにて見つけて
killします.\\

\chapter{TiledDisplayDriverの選択}
TDDは、以下の2つのコントローラ(Display, Controller制御)側か、
Display側かを決定することができます.\\

\chapter{TiledDisplayDriverへのアクセス}
\section{概要}
TiledDisplayDriverのアクセスは、Webブラウザのアドレス欄に
「http://localhost:8080」と入力することでアクセス出来ます.

アクセスすると、以下の画面が表示されます.\\
★画面のスクリーンショット\\


\section{操作}
ここで、アクセスしたPCを「コントローラ」として使用するか、
ディスプレイとして使用するかを選択することができます.\\


\chapter{コントローラ画面の操作}
\section{概要}
コントローラは以下の通りとなっております.\\
★画面のスクリーンショット\\

それぞれの機能について解説します.\\

\newpage

\section{コントローラの操作 : 中央}
中央はVirtualDisplaySpaceと呼ばれ、TiledDisplayServerに接続された
ディスプレイの操作、コンテンツの移動、操作、削除等を行う
汎用スペースとなっております.\\

★画面のスクリーンショット\\


\newpage

\section{コントローラの操作 : 左(Displayタブ)}
★画面のスクリーンショット\\
VirtualDisplayと、TDDサーバーに接続されているDisplayの一覧を表示します.\\
コントローラは、このDisplayをVirtualDisplay上に配置し、コンテンツを追加することによって
共有ワークスペースを実現します.\\
Displayは以下の通りマウスドラッグドロップにより、VirtualDisplaySpaceに配置することができます.\\

★画面のスクリーンショット\\

★画面分割数、解像度の事を記載する



以下、4クライアントが接続された環境の例となります.\\
★4クライアントが接続された状態のDisplayタブのスクリーンショット\\を記載する


Displayを正確に区画に配置するための機能として「snap機能」があります.\\
画面右側の以下のボタンとなります.\\

\begin{tabbing}
0123\=01234567890123\=0123456789\kill
\>Free    \> : 自由配置となります.\\
\>Display \> : 分割した区画に沿ってDisplayがスナップするようになります.
\end{tabbing}

またVirtualDisplaySpaceの拡大縮小オプションとして、Scale機能があります.\\
画面右側の以下のボタンとなります.\\
デフォルトは0.5となっております.\\

\subsection{Show Display ID}
接続されたDisplayのIDを各接続されたDisplay上に表示し、識別できるようにします.\\
尚、IDは、接続された端末固有であり、1端末につき1IDが割り当てられます.\\

\subsection{Delete}
選択したDisplayを削除(TDDサーバーから切断)します.\\
尚、VirtualDisplayは削除することはできません.\\



\subsection{DeleteAll}
接続されているDisplayすべてを削除(TDDサーバーから切断)します.\\
尚、VirtualDisplayは削除することはできません.\\

\newpage

\section{コントローラの操作 : 左(contentsタブ)}

★画面のスクリーンショット\\

各ボタンの機能は以下の通りとなります.\\

\subsection{Addボタン}
コンテンツの追加を行います.\\
押下することで、AddContentウィンドウを開きます



\subsection{Deleteボタン}
選択されたコンテンツを削除します.\\

\chapter{コントローラの操作 : AddContentウィンドウ}
\subsection{テキストの送信}
入力されたテキストをコンテンツに追加します.\\
以下追加例となります.\\
\\

★テキスト追加のスクリーンショット\\


\subsection{テキストファイルの送信}
テキストファイルをコンテンツに追加します.\\
以下追加例となります.\\

★テキストファイル例のスクリーンショット\\
★テキストファイル追加したスクリーンショット\\


\subsection{URLの送信}
入力されたURLのサイトの画像をコンテンツに追加します.\\
以下例となります.\\
★URL画像追加のスクリーンショット\\追加



\subsection{画像の送信}
任意の画像ファイルをコンテンツに追加します.\\
対応している画像フォーマットは以下の通りです.\\
-- png, jpeg
以下に表示例となります.\\
★画面のスクリーンショット\\


\subsection{画像の差し替え}
contentsタブにて選択している画像の差し替えを行います.\\
差し替え例を以下に示します.\\
★画面のスクリーンショット\\


\chapter{コントローラの操作 : 右 propartyウィンドウ }

propartyウィンドウは選択されたコンテンツ、Display、ContentsのID、
およびそれぞれのpropartyを表示します.\\

propartyは以下の通りID以外を編集し、座標、表示全面の優先順位 Zindex を
指定することができます.\\
以下操作例となります.\\

★画面のスクリーンショット\\


また、選択されたContentsはpropartyウィンドウ左下のダウンロードボタンから
ダウンロードすることができます.\\

★画面のスクリーンショット\\


\chapter{コントローラの操作 : 上}
\subsection{Controllerボタン}
Displayウィンドウを表示します.\\
現状は押下しても特に意味を持ちません.\\

\subsection{Displayボタン}
Displayウィンドウを新しいタブで表示します.\\


\subsection{Virtual Display Settingボタン}
Displayタブに操作をフォーカスします.\\


\end{document}
